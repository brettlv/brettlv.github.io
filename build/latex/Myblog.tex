%% Generated by Sphinx.
\def\sphinxdocclass{report}
\documentclass[letterpaper,10pt,english]{sphinxmanual}
\ifdefined\pdfpxdimen
   \let\sphinxpxdimen\pdfpxdimen\else\newdimen\sphinxpxdimen
\fi \sphinxpxdimen=49336sp\relax

\usepackage[margin=1in,marginparwidth=0.5in]{geometry}
\usepackage[utf8]{inputenc}
\ifdefined\DeclareUnicodeCharacter
  \DeclareUnicodeCharacter{00A0}{\nobreakspace}
\fi
\usepackage{cmap}
\usepackage[T1]{fontenc}
\usepackage{amsmath,amssymb,amstext}
\usepackage{babel}
\usepackage{times}
\usepackage[Bjarne]{fncychap}
\usepackage{longtable}
\usepackage{sphinx}

\usepackage{multirow}
\usepackage{eqparbox}

% Include hyperref last.
\usepackage{hyperref}
% Fix anchor placement for figures with captions.
\usepackage{hypcap}% it must be loaded after hyperref.
% Set up styles of URL: it should be placed after hyperref.
\urlstyle{same}
\addto\captionsenglish{\renewcommand{\contentsname}{Contents:}}

\addto\captionsenglish{\renewcommand{\figurename}{Fig.\@ }}
\addto\captionsenglish{\renewcommand{\tablename}{Table }}
\addto\captionsenglish{\renewcommand{\literalblockname}{Listing }}

\addto\extrasenglish{\def\pageautorefname{page}}

\setcounter{tocdepth}{1}



\title{Brettlv's blog}
\date{Sep 02, 2019}
\release{3}
\author{Brett}
\newcommand{\sphinxlogo}{}
\renewcommand{\releasename}{Release}
\makeindex

\begin{document}

\maketitle
\sphinxtableofcontents
\phantomsection\label{\detokenize{index::doc}}


\href{https://brettlv.github.io/}{我的Github主页}

欢迎访问~


\chapter{入门篇}
\label{\detokenize{base/index:welcome-to-brettlv-s-blog}}\label{\detokenize{base/index::doc}}\label{\detokenize{base/index:id1}}

\section{基础}
\label{\detokenize{base/index:id2}}
这一部分主要介绍数据科学的入门内容;包含数据科学的基础工具,如:Jupyter、Linux,以及 Python 基本的数据科学包 Numpy,画图包 Matplotlib;


\subsection{Linux}
\label{\detokenize{base/01_linux::doc}}\label{\detokenize{base/01_linux:linux}}
linux 简单命令

ls \#show file list

pwd \#directory

cd . \#load in dir

cd .. \#back in dir

mkdir \#make dir

rm (-rf) \#delete file/dir

sh test.sh \#run file

vim test.py \#edit file


\subsection{python}
\label{\detokenize{base/02_ipython:python}}\label{\detokenize{base/02_ipython::doc}}\label{\detokenize{base/02_ipython:vim}}

\subsection{numpy}
\label{\detokenize{base/03_numpy:numpy}}\label{\detokenize{base/03_numpy::doc}}

\subsection{plot}
\label{\detokenize{base/04_matplotlib:plot}}\label{\detokenize{base/04_matplotlib::doc}}

\chapter{高级篇}
\label{\detokenize{advanced/index::doc}}\label{\detokenize{advanced/index:id1}}

\section{进阶}
\label{\detokenize{advanced/index:id2}}
这一部分主要介绍数据科学的进阶内容;包含数据科学常用包的工具,如:
pycbc,gwpy,astropy,mcmx;


\subsection{pycbc}
\label{\detokenize{advanced/01_pycbc:pycbc}}\label{\detokenize{advanced/01_pycbc::doc}}
pip install lalsuite pycbc


\subsection{gwpy}
\label{\detokenize{advanced/02_gwpy::doc}}\label{\detokenize{advanced/02_gwpy:gwpy}}
pip install gwpy


\subsection{mcmc}
\label{\detokenize{advanced/03_mcmc:mcmc}}\label{\detokenize{advanced/03_mcmc::doc}}

\subsection{astropy}
\label{\detokenize{advanced/04_astropy:astropy}}\label{\detokenize{advanced/04_astropy::doc}}
anaconda2\&3


\chapter{曜灵篇}
\label{\detokenize{poem/index::doc}}\label{\detokenize{poem/index:id1}}

\section{序}
\label{\detokenize{poem/Preface::doc}}\label{\detokenize{poem/Preface:id1}}

\subsection{开篇之语}
\label{\detokenize{poem/Preface:id2}}
曜灵集序


\section{梦}
\label{\detokenize{poem/Chap1::doc}}\label{\detokenize{poem/Chap1:id3}}\label{\detokenize{poem/Chap1:id1}}

\subsection{好梦安好}
\label{\detokenize{poem/Chap1:id2}}
梦


\section{寻}
\label{\detokenize{poem/Chap2::doc}}\label{\detokenize{poem/Chap2:id3}}\label{\detokenize{poem/Chap2:id1}}

\subsection{寻寻世界}
\label{\detokenize{poem/Chap2:id2}}
寻


\section{悟}
\label{\detokenize{poem/Chap3::doc}}\label{\detokenize{poem/Chap3:id3}}\label{\detokenize{poem/Chap3:id1}}

\subsection{若有所思}
\label{\detokenize{poem/Chap3:id2}}
悟


\section{尾}
\label{\detokenize{poem/END::doc}}\label{\detokenize{poem/END:id3}}\label{\detokenize{poem/END:id1}}

\subsection{渺渺之言}
\label{\detokenize{poem/END:id2}}
序尾


\chapter{Indices and tables}
\label{\detokenize{index:indices-and-tables}}\label{\detokenize{index:id3}}\begin{itemize}
\item {} 
\DUrole{xref,std,std-ref}{search}

\end{itemize}



\renewcommand{\indexname}{Index}
\printindex
\end{document}