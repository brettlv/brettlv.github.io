%% Generated by Sphinx.
\def\sphinxdocclass{report}
\documentclass[letterpaper,10pt,english]{sphinxmanual}
\ifdefined\pdfpxdimen
   \let\sphinxpxdimen\pdfpxdimen\else\newdimen\sphinxpxdimen
\fi \sphinxpxdimen=49336sp\relax

\usepackage[margin=1in,marginparwidth=0.5in]{geometry}
\usepackage[utf8]{inputenc}
\ifdefined\DeclareUnicodeCharacter
  \DeclareUnicodeCharacter{00A0}{\nobreakspace}
\fi
\usepackage{cmap}
\usepackage[T1]{fontenc}
\usepackage{amsmath,amssymb,amstext}
\usepackage{babel}
\usepackage{times}
\usepackage[Bjarne]{fncychap}
\usepackage{longtable}
\usepackage{sphinx}

\usepackage{multirow}
\usepackage{eqparbox}

% Include hyperref last.
\usepackage{hyperref}
% Fix anchor placement for figures with captions.
\usepackage{hypcap}% it must be loaded after hyperref.
% Set up styles of URL: it should be placed after hyperref.
\urlstyle{same}
\addto\captionsenglish{\renewcommand{\contentsname}{Contents:}}

\addto\captionsenglish{\renewcommand{\figurename}{Fig.\@ }}
\addto\captionsenglish{\renewcommand{\tablename}{Table }}
\addto\captionsenglish{\renewcommand{\literalblockname}{Listing }}

\addto\extrasenglish{\def\pageautorefname{page}}

\setcounter{tocdepth}{1}



\title{Myblog Documentation}
\date{Aug 09, 2018}
\release{3}
\author{Brett}
\newcommand{\sphinxlogo}{}
\renewcommand{\releasename}{Release}
\makeindex

\begin{document}

\maketitle
\sphinxtableofcontents
\phantomsection\label{\detokenize{index::doc}}



\chapter{入门篇}
\label{\detokenize{base/index:welcome-to-myblog-s-documentation}}\label{\detokenize{base/index::doc}}\label{\detokenize{base/index:id1}}
这一部分主要介绍数据科学的入门内容;包含数据科学的基础工具,如:Jupyter、Linux,以及 Python 基本的数据科学包 Numpy,画图包 Matplotlib;


\section{Linux basement}
\label{\detokenize{base/01_linux::doc}}\label{\detokenize{base/01_linux:linux-basement}}

\section{numpy}
\label{\detokenize{base/03_numpy:numpy}}\label{\detokenize{base/03_numpy::doc}}

\section{plot}
\label{\detokenize{base/04_matplotlib:plot}}\label{\detokenize{base/04_matplotlib::doc}}

\chapter{高级篇}
\label{\detokenize{advanced/index::doc}}\label{\detokenize{advanced/index:id1}}
这一部分主要介绍数据科学的进阶内容;包含数据科学常用包的工具,如:
pycbc,gwpy,astropy,mcmx;


\chapter{Indices and tables}
\label{\detokenize{index:indices-and-tables}}\label{\detokenize{index::doc}}\begin{itemize}
\item {} 
\DUrole{xref,std,std-ref}{genindex}

\item {} 
\DUrole{xref,std,std-ref}{modindex}

\item {} 
\DUrole{xref,std,std-ref}{search}

\end{itemize}



\renewcommand{\indexname}{Index}
\printindex
\end{document}